\begin{abstract}
Swarm testing is an approach to improving the effectiveness of random
testing that replaces the traditional testing procedure of generating
tests from a single probability distribution with the use of a
``swarm'' of distributions each of which disallows certain test
behaviors completely.  This paper shows that swarm has two highly
desireable properties.  First, swarm is in many cases extremely
effective, resulting in 40\% or better improvements in \emph{distinct
faults detected} for critical real-world systems such as compilers.
Second, swarm is an encourgagingly \emph{lightweight} method for
improving random testing, with very few demands on would-be users;
applying swarm to an existing random testing system is almost always
trivially easy. In our experience, even a programmer unfamiliar with a
complex testing system can often apply swarm testing to it.  The
widespread applicability and effectiveness of swarm testing suggests a
new domain of random testing research: the search for techniques that
rely only on general features of most test spaces, and do not require
much additional programmer effort or analytical machinery to apply to
already existing random testing frameworks, which are frequently
complex and difficult to modify.  This paper presents case studies
demonstrating the lightweight nature and high effectiveness of swarm
across a set of important real-world case studies.
%
\end{abstract}

