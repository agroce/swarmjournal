%% -*- mode: LaTeX -*-
%%

%%%%%%%%%%%%%%%%%%%%%%%%%%%%%%%%%%%%%%%%%%%%%%%%%%%%%%%%%%%%%%%%%%%%%%%%%%%%%%%

\section{Conclusion}

Swarm testing relies on the following claim: for realistic systems,
\emph{randomly excluding some features from some tests} can improve
coverage and fault detection, compared to a test suite that
potentially uses every feature in every test.
%
The benefit of using of a single, inclusive, default
configuration---that every test can potentially expose any fault and
cover any behavior, heretofore usually taken for granted in random
testing---does not, in practice, make up for the fact that \emph{some
features can, statistically, suppress behaviors.}  Effective testing
therefore may require feature omission diversity.
%
We show that this not only holds for simple container-class examples
(e.g., pop operations suppress stack overflow) but for a widely used
flash file system and 14 out of 17 versions of five production-quality C
compilers.
%
For these real-world systems, if we compare testing with a single
inclusive configuration to testing with a set of 100--1,000
unique configurations, each omitting features with 50\% probability
per feature, we have observed (1)~significantly better fault
detection, (2)~significantly better branch and statement coverage, and
(3)~strictly superior mutant detection.  
%
Test configuration diversity does indeed produce better testing in
many realistic situations.


Swarm testing was inspired by swarm verification, and we hope that its
ideas can be ported back to model checking.
%
We also plan to investigate swarm in the context of bounded exhaustive
testing and learning-based testing methods.
%
Finally, we believe there is room to better understand \emph{why}
swarm provides its benefits, particularly in the context
of large, idiosyncratic SUTs such as compilers, virtual machines, and
OS kernels.
%
More case studies will be needed to generate data to support this
work.  We also plan to investigate how swarm testing's increased
diversity of code coverage in test cases can benefit fault
localization and program understanding algorithms
relying on test cases~\cite{Tarantula}; traditional random tests are far more
homogeneous than swarm tests.


We have made Python scripts supporting swarm testing available at
\url{http://beaversource.oregonstate.edu/projects/cswarm/browser/release}.
%
%% ENE: cut for space.
%
% The {\tt genconfigs} tool supports Gaussian distributions of numeric
% parameters and feature dependencies as well as the binary
% inclusion/omissions used in this paper.

%%%%%%%%%%%%%%%%%%%%%%%%%%%%%%%%%%%%%%%%%%%%%%%%%%%%%%%%%%%%%%%%%%%%%%%%%%%%%%%

%% End of file.
